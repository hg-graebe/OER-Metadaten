\documentclass{beamer}
\usepackage{ngerman,url}
\usepackage[utf8]{inputenc}
\parskip1em

\usetheme{}
\beamertemplatenavigationsymbolsempty

\author{Hans-Gert Gräbe}

\institute{Institut für Informatik der
  Univ. Leipzig\\ \url{http://bis.informatik.uni-leipzig.de/HansGertGraebe}}

\title{ELMAT als OER -- die Metadaten-Perspektive\vskip1em}

\subtitle{Vortrag auf dem 5. Treffen des Netzwerks\\ Mathematik/Physik und
  E-Learning, HTW Dresden}

\date{30. September 2015}
\begin{document}
\begin{frame}
  \maketitle
\end{frame}

\begin{frame}{Open Educational Resources}
Das Konzept der OER – \emph{Open Educational Resources} – gewinnt zunehmend an
Aufmerksamkeit und Bedeutung. Der Begriff wurde erstmals vom \emph{UNESCO 2002
  Forum on the Impact of Open Courseware for Higher Education in Developing
  Countries} verwendet.

Mit diesem Ansatz wird insbesondere der Gedanke einer einfachen Nachnutzung und
Anpassung verbunden, die drei wesentliche Aspekte zu berücksichtigen hat:
\begin{itemize}
\item rechtliche Aspekte -- Open,
\item inhaltliche Aspekte -- Qualität der Educational Resource,
\item Aspekte der Verfügbarkeit und Indexierbarkeit -- die Metadaten-Perspektive.
\end{itemize}
\end{frame}

\begin{frame}{ELMAT Metadaten -- LOM}

ELMAT-Aufgaben werden in einer zip-Datei ausgeliefert, die ein IMS Manifest
enthält sowie die eigentlichen Aufgabenressourcen als XML-Dateien in einem
Format, das im Onyx-Player abgespielt werden kann.

Das IMS Manifest enthält Metainformationen zur Aufgabe im LOM-Format (Learning
Object Metadata) 
\begin{center}\small
  \url{https://en.wikipedia.org/wiki/Learning_object_metadata}
\end{center}
LOM ist im Kern ein Datenmodell für Metadaten von Lerner-Objekten und ein
IEEE-Standard, der zusammen mit dem IMS erarbeitet wurde.  Die aktuelle Version
ist aus dem Jahr 2002.  
\end{frame}

\begin{frame}{LOM-Bindings}

Für digitale Zwecke muss der Standard an Beschreibungssprachen gebunden werden.
Es sind Bindings an XML (IEEE 1484.12.3) und RDF (IEEE 1484.12.4) spezifiziert.

Die beiden Bindings unterscheiden sich in ihrer Aussagekraft deutlich. XML ist
eine Markup Language und vom Sprachkonzept her nur dafür entworfen,
\emph{syntaktische} Auszeichnungen vorzunehmen.  RDF ist entworfen, um
\emph{semantische} Informationen auf einheitliche Weise darzustellen, die in
XML in je spezifischer Weise in einem XSchema kodiert werden können. 

ELMAT/Onyx setzt auf dem XML-Binding von LOM auf und kann damit semantische
Informationen nur informell oder auf speziell zu vereinbarende Weise innerhalb
des verwendeten LOM-Dialekts darstellen. 
\end{frame}

\begin{frame}{LOM, XML, RDF und Dublin Core}
2002--2005 gab es intensive Versuche, einen XML-RDF-Parser für einzelne
LOM-Dialekte zu entwerfen, um die im Manifest vorhandenen semantischen
Informationen in ein allgemein verständliches Sprachformat zu bringen.

Mit der Festigung und Etablierung von DC (Dublin Core) und der zunehmenden
Bedeutung der DCMI (Dublin Core Meta Initiative) als übergreifendes Format und
Projekt zur Herstellung von Interoperabilität im Bereich der Metadaten zu
digitalen Ressourcen, insbesondere mit den weltweiten
Standardisierungsprojekten um „Resource Description and Access“ (RDA) im
Bereich der Bibliotheken wurden diese Versuche weitgehend eingestellt. 

LOM ist damit zur Herstellung von Interoperabilität auf Metadaten-Ebene in
einer Linked Open Data Welt nur sehr bedingt geeignet.
\end{frame}

\begin{frame}{Metadaten auf RDF-Basis}
RDF bedeutet \emph{Resource Description Framework} und stellt darauf ab, dass
es sinnvoll ist, eine deutliche Trennung von Ressourcen (Daten) und
Ressourcen-Beschreibungen (Metadaten) vorzunehmen, da letztere eine
eigenständige Bedeutung sowohl in technischer als auch sozialer Hinsicht haben,
vergleichbar dem Unterschied zwischen den Büchern einer Bibliothek und dem
Katalog.  

Ressourcen können verteilt verwaltet werden, Metadaten müssen zusammengeführt
werden (können), um eine Gesamtsicht zu triggern. Hierfür haben sich
\emph{Linked Open Data} Prinzipen und Standards auf RDF-Basis bewährt. 
\end{frame}

\begin{frame}{ELMAT Metadaten-Repository -- prinzipielle Überlegungen}
Es wäre deshalb spannend, auf diesen Standards ein Metadaten-Repository zu
ELMAT aufzubauen, das unabhängig von der Onyx-Realisierung der eigentlichen
Ressourcen ist, nach Linked Open Data Standards entworfen ist und von einem
offenen Community-Prozess zur Qualitätssicherung der Aufgaben begleitet wird. 

Ein diesbezügliches Proposal als Teilprojektskizze zum Schwerpunkt „Offene
Infrastrukturen und Lehr-/Lernsysteme“ zum Aufruf zur Bildung von
Projektkonsortien für Verbundvorhaben in strategischen Handlungsfeldern im
Rahmen der Initiative „Bildungsportal Sachsen“ in den Jahren 2015 und 2016 kam
allerdings nicht einmal in die engere Auswahl.
\end{frame}

\begin{frame}{ELMAT Metadaten-Repository -- eine erste Skizze}
In einer ersten Skizze wurde ein Dump der ELMAT-Aufgaben vom Februar 2015 (933
Aufgaben) analysiert, aus den IMS Manifesten ein Teil der Metadaten extrahiert,
in RDF umgewandelt und im RDF-Store 
\begin{center}
  \url{http://pcai003.informatik.uni-leipzig.de/kosemnet/} 
\end{center}
als Kern der Metadaten eines OER-Portals nach Open Data Standards (zunächst als
Unterprojekt im SymbolicData-Kontext) verfügbar gemacht.
\end{frame}

\begin{frame}{ELMAT Metadaten-Repository -- eine erste Skizze}
Der Datenbestand kann damit mit SPARQL-Anfragen an den Endpunkt 
\begin{center}
  \url{http://pcai003.informatik.uni-leipzig.de:8893/sparql}
\end{center}
untersucht werden. Bereits eine erste oberflächliche Untersuchung zeigt die
geringe Qualität der aktuell gespeichertn Metadaten, die kaum kuratiert (selbst
die Autorendaten sind sehr heterogen) und damit wenig für eine strukturierte
Vernetzung geeignet sind. Dasselbe gilt für die Verschlagwortung mit
Schlüsselworten.

Das Konzept eines solchen OER-Portal mit Metainformationen zu verfügbaren
OE-Ressourcen lässt sich leicht auf andere OER-Sammlungen erweitern und hat
sich bereits in den Projekten \emph{SymbolicData} und \emph{KoSemNet} bewährt. 
\end{frame}

\begin{frame}{edu-sharing}
  \begin{itemize}
  \item Netzwerk, das sich seit 2013 entwickelt
  \item Primär auf die Verwaltung von Ressourcen gerichtet, nicht von
    Ressourcenbeschreibungen 
  \item Suche nach „edu share linked open data“ -- keine sinnvollen Treffer 
  \item Keinerlei Verbindung zu den Bemühungen einer Verlinkung auf
    Metadatenebene der deutschen (und internationalen) Bibliotheken wie etwa
    RDA (\url{http://www.dnb.de/rda})
  \end{itemize}

  Im Rahmen der Internationalisierung der deutschen Standards beteiligt sich
  die Arbeitsstelle für Standardisierung der DNB im Auftrag des
  Standardisierungsausschusses aktiv am Entstehungsprozess des neuen
  internationalen Standards „Resource Description and Access (RDA)“.
\end{frame}

\begin{frame}{OER-Metadatenökologie}

„Auf dem Weg zu einer offenen OER-Metadatenökologie“ (Untersuchungen von Adrian
  Pohl, Stand 2013)

Ergebnisse dort auf einem Etherpad zusammengefasst

Siehe auch seinen kompletten Bericht zur OER-Konferenz 2013 auf 
  \url{https://wiki1.hbz-nrw.de/pages/viewpage.action?pageId=10453369}

Kompetenzzentrum Interoperable Metadaten (KIM), eine DINI AG
\url{https://wiki.dnb.de/pages/viewpage.action?pageId=43523047}

Dort gibt es auch eine OER-Metadaten-Gruppe, Leitung Adrian Pohl
\url{https://wiki.dnb.de/display/DINIAGKIM/OER-Metadaten-Gruppe}
\end{frame}
\end{document}

http://od.fmi.uni-leipzig.de:8893/sparql

PREFIX at: <http://symbolicdata.org/autotool/Model/> 
select ?t ?l ?pn ?np ?ne
from <http://symbolicdata.org/autotool/Configurations/>
where {
 ?a a at:Course; rdfs:label ?l; at:designedBy ?p ; at:hasTitle ?t .
optional {?a at:numberOfExams ?ne .} 
optional {?a at:numberOfParticipants ?np . }
 ?p foaf:name ?pn .
}
